\chapter{Hierrachical Quadratic Programming}
\label{chapter-4}

In robotics, the use of inverse kinematics is very common and widely implemented for any applications. In humanoid robots, the robot structure
is redundant which means it has more DoF than the number need to position the end-effector in the desired pose. These extra DOFs can be used to
perform other tasks like \textit{avoiding singularities, balancing, dodging obstacles, minimizing energy consumption, etc.} 

Usually more than one aspect of the task must be transmitted to the robot. For this purpose, \textit{task prioritization} will be used by allowing the jacobian of one task 
into the null projection of another task \cite{khalil2004modeling}. To implement the \textit{task prioritization} effectively, an highly efficient method known as 
"Hierrachical Quadratic Programming (HQP)" can be used. HQP results in handling stacks of tasks with less computation intensity \cite{escande2014hierarchical} . This chapter details the task specification and HQP for this
research from the concepts explained in chapter \ref{chapter-3}.

\section{Types of Tasks}

A task or constraint refers to any geometric goals which must be achieved by the robot. For example, reaching a certain point in the world frame, reaching a 
particular joint angle, avoiding an obstacle, placing the center of mass in a specific location and so on. Another way of seeing them is as constraints 
which limit the robot’s allowed motion: the robot can place itself in any way as long as it doesn’t disturb the task.

Several tasks of each type will be performed. Each type will be described separately below. The mathematical description will be given as well as 
examples of when that type of task is useful. The way of arranging these tasks according to their priorities will be detailed. And finally, the tasks 
used in this work will be defined.

\subsection{Equality tasks}

This type of task or constraint is the most commonly found when dealing with inverse kinematics. It consists of tracking values which can be described in function 
of the robot’s joints. For example, the position and orientation of an end-effector, the position of the center of mass, and many other key points in 
the robot’s body can be described in function of the joint configuration. For a given task involving \textit{DGM} and \text{IGM}, let the pose descriptor be $X$ and joint descriptor be $q$. Then the change in joint vector $\Dot{q}$ with jacobian $J$
 is given by,

 \begin{equation}
     \label{eq: equality-task}
    \dot{X} = J \dot{q}
 \end{equation}

Then the objective that need to be accomphlished can be represented in the task space velocity as $\dot{e}_H^*$. Once obtained, the robot input can be computed as a quadratic problem as,
\begin{equation}
    \label{eq: equality-task-2}
    \mathit{Find} \quad \dot{q}^*  \in \min_{\dot{q}} ||J\dot{q} - \dot{e}_H^*|| \\
\end{equation}

More generally, the linear constraints in equation \ref{eq: equality-task-2} can be represented in the form $Ax = b$ as \cite{escande2014hierarchical},

\begin{equation}
    \label{eq: equality-task-3}
    \mathit{Find} \quad x^*  \in \min_{x} ||Ax - b|| \\
\end{equation}

Among the possible task vector $x^*$, the pseudo inverse can be computed as ,

$$x = A^+b$$

$A^+$ can be calculated from Singular Value Decomposition (SVD).
This simplifies the calculation, since the pseudo-inverse of a diagonal matrix is found by taking the reciprocal of each of its non-zero elements \cite{louisepouble}.

\begin{align}
    \label{eq: SVD}
    \begin{split}
        A^+_L &= (A^tA)^{-1}A^t \\
        A^+_R &= A^t(AA^t)^{-1}     
    \end{split}
\end{align}

From equation \ref{eq: SVD}, the pseudo inverse can be calculated with less computation with linear independent columns or independent rows \cite{louisepouble}. 
With these equality tasks, the cartesian tracking, CoM projection computation for balance, avoiding singularities and much mor can be achieved.

\subsection{Inequality tasks}

This type of tasks or constraints are most commonly used for bound type functions like joint limit, joint velocity limit and much more. It consists of maintaining values 
within the theshold limit Similar to equality-only least square quadratic program in equation \ref{eq: equality-task-3}, a set of linear inequality constraints can be 
written as,

\begin{equation}
    \label{eq: inequal-task-1}
    \mathit{Find} \quad x^* \in \{ x, \ \mathit{s.t} \  Cx \le d\}
\end{equation}

The above constraints are very useful, for example, for minimizing the diverence between captured human joint angles and the robot’s angles. It is also commonly used in industrial robots to maximize the distance from the joint angles and the joint limits. Unlike equality tasks, optimization tasks cannot be hard constraints, since the optimal value might not be reached.

\section{Quadratic Programming Approach}

Humanoid robots are usually redundant with respective allowing to add more tasks to the task space. Ensuring task priorities for the assigned tasks 
will help the optimum performance. However, to achieve a task in three-dimensional space, will require infinite joint configurations to achieve a goal.
The space where motions can be performed without perturbating the task is known as task's null projector.
To efficiently compute and optimize the tasks for the application, hierrachical quadratic programming can be used as proposed in \cite{escande2013planning}. This
optimization technique provides  online faster computation and can cope up with the redundancy of the humanoid robot.


\subsection{Null Space Projection}

The null space projector is directly linked to task Jacobian and is alternatively called as Jacobian's null projector. The act of performing a task within the other task is usually called as
projecting the task vector $Z_2$ into the Jacobian $J_1$'s null space. Then the projection on Jacobian $J_1$ is given by,

\begin{equation}
    \label{eq: null-space-1}
    P_1 = I - A_1^+ A_1
\end{equation}

where $I$ is the identity matrix. The task prioriy is set such that task $j$ has higher prioirity than task $j+1$. Then equation \ref{eq: equality-task} can be computed as quadratic problem as,

\begin{equation}
    \label{eq: null-space-2}
    \min_x(\underline{A}_p x - \underline{b}_p)^T Q (\underline{A}_p x - \underline{b}_p) 
\end{equation}

such that,
\begin{equation*}
    \underline{A}_p = \begin{bmatrix}
        A_1 \\ A_2 \\ ... \\ A_p
    \end{bmatrix} \qquad 
    \underline{b}_p = \begin{bmatrix}
        b_1 \\ b_2 \\ ... \\ b_p
    \end{bmatrix}
\end{equation*}


where $(\underline{A}_p, \underline{b}_p)$ is a set of non-conflicting linear constraints that need to be satisfied for best simultaneously and $Q$ is the weighing matrix for optimizing the constraints for the tasks proposed.
Then the  second contraint $(A_2, b_2)$ is solved with the null projector of task 1 with constraint $(A_1, b_1)$ using the minimization of quadratic programming as \cite{escande2014hierarchical},

\begin{equation}
    \label{eq: null-space-3}
    \min_x ||A_2P_1x_2 - (b_2 - A_2A_1^+b_1)||
\end{equation}

Similarily the third constraint $(A_3, b_3)$ on the null projection of task 2 can be given by,

\begin{equation}
    \label{eq: null-space-4}
    x_2^* = (A_2P_2)^+(b_2 -  A_2A_1^+b_1) + P_2 \tilde{x}_3
\end{equation}

and the complete solution solving $(A_1, b_1)$ at its best and $(A_2, b_2)$ if possible is given by,

\begin{equation}
    \label{eq: null-space-5}
    x_2^* = A_1^+b_1 + (A_2P_2)^+(b_2 -  A_2A_1^+b_1) + P_2 \tilde{x}_3
\end{equation}

The solution in equation \ref{eq: null-space-5}  can be extended recursively to solve the $p$ levels of the hierrachy with $P_0 = I$, $x_0 = 0$ and $P_k = P_{k - 1}\tilde{P}_k$ the projector into the null space of $A_k$
is given by,

\begin{equation}
    \label{eq: null-space-6}
    x_p^* = \Sigma_{k=1}^p (A_kP_{k-1})^+(b_k -  A_kA_{k - 1}^+b_{k - 1}) + P_k \tilde{x}_{k+1}
\end{equation}

The basic projector $P$ in equation \ref{eq: null-space-6} can be replaced by the advanced projector suggested by Escande \cite{escande2010fast} as,

\begin{equation}
    \label{eq: null-space-7}
    x_p^* = \Sigma_{k=1}^p Z_{k - 1}(A_kZ_{k-1})^+(b_k -  A_kA_{k - 1}^+b_{k - 1}) + Z_pz_{p + 1}
\end{equation}

where $Z_k$ is a basis of a null space of $\underline{A}_k$ (\textit{i.e.} $A_iZ_i = 0$ and $Z_i^TZ_i = I$, $i = 1, 2, ..., k$) and is a vector of dimension of null space of $\underline{A}_k$.
The writing in last equation is more efficient than equation \ref{eq: null-space-6} due to its corresponding matrix size \cite{escande2014hierarchical}.

\subsection{Hierrachy of Equality Quadratic Program}

The Equality-only Hierrachical Quadratic Program is considered (eHQP) as a least-square problem. It is written
as a set of $p$ eHQP: at level k, the QP is  solved using,

\begin{align}
    \label{eq: HQP}
    \quad min_{x_k, w_k}||w_k|| \\
    \mathit{s.t.} \quad    A_kx_k &= b_k + w_k \\
    \underline{A}_{k - 1}x_k &= \underline{b}_{k - 1} + \underline{w}_{k - 1}^*
\end{align}

where $\underline{A}_{k - 1}, \underline{b}_{k - 1}$ and $\underline{w}_{k - 1}^*$ are the matrix and vectors composed of the stacked quantities 
as represented in equation \ref{eq: null-space-2}. The weight $\underline{w}_{k - 1}^*$ is a fixed value updated from the previous QP.
The langrangian of the above equations give the optimality conditions for computing the weights $\underline{w}_k$ 
recursively \cite{escande2014hierarchical}.

\begin{align}
    \label{eq: HQP-2}
    w_k &= A_kx_k - b_k \\
    \underline{A}_{k - 1}x_k &= \underline{b}_{k - 1} + \underline{w}_{k - 1}^ *
\end{align}

The hierrachy is kept such that $(k -1) \gg k \gg (k + 1)$ to prioritize the planned constraints. Equation \ref{eq: HQP-2} confirms that 
higher level constraints are kept when solving for lower level tasks. The eHQP problem derived in this section allows equality constraints to
 be achieved at its best if possible.  The tasks to be solved using eHQP are presented in section \ref{sec: HQP-imitation}.

\subsection{Hierrachy of Inequality Quadratic Program}

As a minimization HQP problem, the linear inequality constraints in equation \ref{eq: inequal-task-1} for two tasks $(p = 2)$ is formulated as,

\begin{eqnarray}
    \label{eq: inequal-task-2}
    \min_{x, w_2} ||w_2|| \\
    \mathit{subject \ to} \ C_{1}x \le d_1 + w_1^* \\
    C_2x \le d_2 + w_2
\end{eqnarray}

This iHQP problem can be solved using a classical active search. However, this would not take into account the specific form of the varibable $w_2$
In practice, its specific role in the constraint equations can be taken into account to reduce the amount of computation.
Concluding the HQP problem containing both equality and inequality constraints, the formulation can be written as,

\begin{align}
    \label{eq: HQP-final}
    \dot{q}_i = \min_{\dot{q},w_i} ||A_i\dot{q} - b_i||^2 + ||w||^2 \\
    \mathit{s.t.} \ C_i\dot{q} - w \le d_i \\
    \mathit{s.t.} \ C_{i - 1}\dot{q} \leq d_{i-1} \\
    \mathit{s.t.} \ A_{i - 1}\dot{q} = b_{i - 1}
\end{align}

\section{HQP in Motion Imitation}
\label{sec: HQP-imitation}

When imitating human motion, it is necessary to first clarify exactly which aspects of the motion are to be copied. Among these, it is also necessary to specify which ones 
are the most important. Is tracking the hand position in the world frame more important than tracking the elbow angle? Does the robot need to keep balance or is it attached 
to some structure? The answers to these questions will influence in the quantity, type and priority of the tasks chosen.

In this work, the following tasks will be considered: \textit{joint-limit avoidance, end-effector tracking, joint tracking, ZMP balancing, posture tracking and fall-detection}.
Each task will be discussed individually.

\subsection[Joint Limit Avoidance]{Task 1: Joint Limit Avoidance}
To imitate the motion, staying away from joint limits should be a high priority of all. In fact, in any robotics applications, avoiding the joint limits
is always the highest prioirity. The solution achieved from imitation should stay within the joint limit configurations and the resultant joint configuration 
shouldn't comply other tasks. 

Staying away from the joint limits can be described as staying as close as possible to the median value for the joint position.
The joint limit avoidance defines a bound on a function of the robot configuration:

\begin{equation}
    \label{eq: joint-limit-1}
    e_{min}(\Omega) \le e(q) \le e_{max}(\Omega)
\end{equation}

where $q$ and $\Omega$ represent the current joint vector and the desired parameters of the robot motor limits. 
This linear inequality constraint is homogenous to the configuration and can be achieved by,

\begin{equation}
    \label{eq: joint-limit-2}
    [e_{min}(\Omega) - e(q)] \le J\dot{q} \le [e_{max}(\Omega) - e(q)]
\end{equation}

where $J$ represent the robot Jacobian, a set of kinematic chains broken from the robot structure.
Then the inequality constraints for the HQP be represented as,

\begin{align}
    \label{eq: joint-limit-3}
    \begin{bmatrix}
        C_1 \\ C_2
    \end{bmatrix} \dot{q} \le 
    \begin{bmatrix}
        d_1 \\ d_2
    \end{bmatrix}
\end{align}

In the above constraints, $(C_1, d_1) = ()$ and $(C_2, d_2) = ()$ are the constraint matrices and vectors of lower and upper joint limits respectively.

\subsection[Balance Control]{Task 2: Balance Control}

The ZMP and CoM retargetting presented on section \ref{sec: balance-control} can be formulated to obtain the optimal correction that 
satisfies the balance condition on the humanoid.

\begin{eqnarray}
    \label{eq: balance-task-1}
    \min_{pZMP} (\dot{P}_{CoM}^{ref} - \dot{P}_{CoM})^T Q (\dot{P}_{CoM}^{ref} - \dot{P}_{CoM}) \\
    \mathit{such \ that} \ \dot{P}_{CoM} = \dot{P}_{CoM}^* + \frac{1}{z^*_{CoM}}(P_{CoM} - P_{ZMP}) \\
    lb_{SP} \le P_{ZMP} \leq ub_{SP}
\end{eqnarray}

$(lb_{SP}, ub_{SP})$ is the lower and upper limit of Support Polygon; $z_{CoM}^*$ is the height of the CoM in previous QP; $Q$ is the weighing matrix
The computation of CoM height using LIPM model helps in better estimation of ZMP position and consequently, a more accurate correction \cite{gucci:hal-01895145}.

\subsection[End-effector tracking]{Task 3: End-effector tracking}

To track the end-effector of the tree-structured robot, both position and orientations are considered. Considering the human actor task vector $X_H$ and the robot task vector $X_R$,
the end-effector tracking can be achieved and the change in joint vector $\dot{q}$ is,

\begin{equation}
    \label{eq: end-effector-tracking-1}
    \dot{q} = J^+(X_H - X_R)   
\end{equation}

The equality constraints can then be computed using for the current joint configuration $q^c$,

\begin{equation}
    \label{eq: end-effector-tracking-2}
    A_{3}(q^c)\dot{q} = (X_H - X_R) = b_3
\end{equation}

where the constraint matrix $A_3(q^c) = J(q^c)$ is the robot jacobian; $X_H$ and $X_R$ are the set of end-effector states respective to human and robot.

\subsection[Joint trajectories tracking]{Task 4: Joint trajectories tracking}

To imitate the approximately exact motion from human actor data, the differene between joint vectors of human actor $q_H$ and humanoid robot $q_R$ need to be minimum resulting in 
inequality linear constraints and is represented as,

\begin{equation}
    \label{eq: joint-tracking-1}
    \min \kappa ||q_H - q_R||^2
\end{equation}

As a iHQP problem, equation \ref{eq: joint-tracking-1} can be rewritten as ,

\begin{equation}
    C_4\dot{q} \le d_4
\end{equation}

where $C_4 = I$ and $d_4 = \kappa(q_H - q_R)^2 $ is the computation performed. $\kappa$ is a custom weight matrix to provide compartively less weight to leg chains.
\subsection[Posture Control]{Task 4: Posture Control}
The human actor and the robot are simplified to the composition of M-DIP model and the posture control can be implemented directly using the acceleration feedback between the 
two subjects. The acceleration control for the motion-retargetting can be implemented as a weighted-constraint based problem \cite{van1985method},

\begin{equation}
    \label{eq: posture-control-1}
    \min ||\ddot{\mathbb{X}}_H - \ddot{\mathbf{X}}_R||
\end{equation}

