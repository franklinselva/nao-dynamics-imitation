\documentclass{thesisreport}


\begin{document}

 \thispagestyle{empty}

\def\lskip{\vspace{0.5cm}}


\begin{tabular}{p{7cm}p{8cm}}
ÉCOLE CENTRALE DE NANTES
&
% EMARO students only
% \raggedleft FIRST YEAR INSTITUTION	
\end{tabular}

\vspace{2cm}

% CORO-IMARO students
\begin{center} \large\sc MASTER CORO-IMARO\\ \normalsize{``CONTROL and ROBOTICS''} \end{center}

% EMARO students
%\begin{center} \large\sc MASTER ERASMUS MUNDUS \\ \normalsize{EMARO+ ``European Master in Advanced Robotics''} \end{center}


\begin{center}
	2020 / 2021\\
	\lskip
	Master Thesis Report % or bibliography report
	\lskip
	
	Presented by \lskip 
	
	Selvakumar Hastham Sathiya Satchi Sadanandam \lskip
	
	On 17 May 2021 \lskip\lskip
	
	{\Large \textbf{Realtime human motion imitation by humanoid robot with balance constraint}}
	
	\vfill

Jury \lskip
		
	\end{center}
	


\begin{tabular}{p{3cm}p{5cm}p{7cm} }
 % President: & Name & Position (Institution) \\ & & \\     % for final defense only (not bibliography)
 Evaluators: & Dr. Olivier Kermognant & Assistant Professor (LS2N, ECN) \\
	      & Name & Position (Institution) \\ 
	      & Name & Position (Institution) \\ & & \\  & & \\ 
  Supervisor(s):  & Dr. Sophie Sakka & Assistant Professor (LS2N, ECN) \\
% EMARO students only
%(EMARO)  & Co-supervisor from M1 & Position, M1 institution 
\end{tabular}

\lskip

\begin{flushleft}
 Laboratory: Laboratoire des Sciences du Numérique de Nantes LS2N
\end{flushleft}

\newpage
\thispagestyle{empty}
\null
\newpage
\addtocounter{page}{-1}
\pagestyle{fancy}
  
 
  \section*{Abstract}
   
  Humanoid robots are made as mirror to the humans. Consequently, the humanoid motion is also expected to be real as 
  human and it is natural to use human motion as an input to generate humanoid motion like a child learning from recogniting
  an action and imitating. This process in robots is called motion imitation and there are several challenges posed 
  due to kinematics and dynamics of the robot over the past decades. Although the work on the kinematic challenges is 
  actively improving and notably better than dynamics, it allows the robot only to move and imitate slow actions. 
  For the fast paced motions, \textit{momentum} gets build up and needs dynamics to be taken into account. Due to the 
  differences in redundancy between humanoid robots and humans, real-time imitation in humanoid robots while keeping 
  balance and support changes is still an interesting problem that need to be addressed.

  ~
  
  On this research, motion retargetting or the motion imitation based on dynamic balance and support with dynamic filtering 
  is planned to be implemented on humanoid robot \textit{NAO v5}, from Aldebaran Robotics. The imitation will be carried out 
  online using marker-based motion capture suit from \textit{Xsens}, specifically \textit{Xsens MVN Analyze} system.
  
  ~
  
  The captured motion from the inertial suit will be preprocessed for its representation in operational space and will be 
  scaled to the robot's dimensions. From this scaled motion, for this thesis, a set of actions are taken and will be addressed
  for the balance problem in \textit{NAO} robot. The joint, CoM and the ZMP data of the human actor will be mapped to the robot
  using the scaling function directly. To ensure the stability and to keep up the speed with the human actor, an additional 
  dynamic filter and multi-inverted pendulum based posture control will be implemented. 
 
 \newpage
 
%  \section*{Acknowledgements}
 
%  I would first like to thank my thesis advisor Dr Sophie Sakka for her motivation and opportunity for the thesis topic.
%   I would like forward my thanks to Prof. Olivier Kermorgant for his continuous motivation and help in informing 
%   students about thesis positions without which I would not have found the position. 

% ~

% I would also like to thank Prof. Ina Taralova for her indispensable advice regarding the rules and practices involved 
% in the organisation of this report , which I have tried to implement to the best of my abilities.

% ~

% Finally, I must express my very profound gratitude to my parents and my friends for providing me with unfailing 
% support and continuous encouragement throughout my years of study and through the process of researching and writing 
% this report. This accomplishment would not have been possible without them.
%  \newpage
 
 
\section*{Notations}
    \begin{tabular}{p{3cm}p{10cm}}
    CoM & Centre of Mass
    \end{tabular}

\newpage

\section*{Abbreviations}

\begin{tabular}{p{3cm}p{10cm}}
CoM & Centre of Mass \\
ZMP & Zero Moment Point \\

\end{tabular}

 \newpage
 
 \tableofcontents

 \listoffigures
 
\listoftables
 

 
 \chapter{Introduction}
%  \addcontentsline{toc}{chapter}{Introduction}	 % non-numbered chapters do not appear in table of contents by default

Humanoid robots are expected to exist and work in a close relationship with human beings in the everyday world and 
to serve the needs of physically handicapped people. These robots must be able to cope with the wide variety of tasks
and objects encountered in dynamic unstructured environments. Imagining an humanoid robot collaborates with humans 
to execute some daily tasks, learn actions from humans and even improve it's ability to teleoperate \cite{FUKAYA2001273}.
When a humanoid robot works in collaboration with human, the interaction through gestures and cooperation is essential. 

~

Besides the satisfaction of human curiosity and imagination, the integration of humanoid robots in our daily lives makes
sense for a variety of practical reasons. Robots resembling us would make human-robot interaction more natural and thus
more intuitive and pleasurable. Moreover, if robots are to assist people in daily chores, they have to fit the human 
environment, which is suitable for \textit{human morphology} \cite{Kemp2008}. Performing tasks with two hands, handling
tools, climbing stairs, reaching shelves - to name only a few tasks - require our assistants to have a similar 
morphology to ours, so robots can adapt to human lives, instead of us having to do the opposite. Furthermore, 
research in humanoid robots directly contributes to the field of prosthesis and exoskeletons.

~

The challenges in creating such machines are numerous. Human bodies are energy efficient machines with extremely elaborated 
mechanics and incredible cognitive and perceptual abilities. Thus, the creation of humanoid robots requires advances in 
more areas than one, from the improvement of sensors and processing of information, to efficient control techniques and 
suitable mechanical structures with constraints in shape, size and weight to improve the resemblance to human beings.

~

In order to develop more autonomous interactions in humanoid robots, one of the major and main objective is that it should
be able to imitate human motions precisely. This would be a major milestone to be achieved in the field of humanoid robots.
Scientists and researchers, over the past decades are thriving to make the humanoid robot movements as close as human 
motions. Recently, the ability of the humanoid robots to teleoperate increases rapidly. To imitate the motion perfectly,
the robot should be able to understand the motion; hence, the importance of motion capture systems has widely increased
in robotics especially in humanoid robots.

~

Nowadays, a variety of technologies exist that allow for high accurate capturing of human motions with high frequency. 
By imitating captured motion, humanoid robots can be teleoperated and also learn new skills resembling human actions. 
However, there is a catch because the direct imitation of captured movements is impossible due to the differences in 
the degree of freedoms and the weight distributions between humans and humanoids. Depending on the complexity of the 
motion, the challenge of motion generation increases due to various humanoid's constraints including the constraints in 
stability and the extended period of imitation. 

\section{Measuring Human Movement}

Kinesiology is defined as the scientific study of human movement. To access human motion, Kinesiology involves principles
and methods from \textit{biomechanics, anatomy, physiology and motor learning}. Its range of application includes health
promotion, rehabilitation, ergonomics, health and safety in industry, disability management, among others. The 
measurement of human movement is one of the tools that is central in this research field. In 19th century, various 
devices were built to produce the moving pictures, among those exists the most advanced technique named 
\textit{chronophotography}. This device allowed to study fast paced human motions by recording and reproducing the 
captured motion \cite{RosenhahnBodo2008HMUM}. Another major contribution in motion study is the study on the path of 
center of mass during human displacement \cite{alma991010593879705596}.

\begin{figure}[h!]
\centering
% \captionsetup{justification=centering,margin=2cm}
\includegraphics[scale=0.4]{images/intro-2.png}\hfill
\caption[Human Locomotion using Differential Equations]{Computation of human locomotion using differential equations by Weber Brothers \cite{alma991010593879705596}. The coordination between arms and legs was observed clearly.}\hfill
\label{motion-coordination}
\end{figure}

The first experimental studies of human gait \cite{alma991010593879705596}, i.e. determining physical quantities 
like inertial properties, were conducted by Christian W. Braune (1831-1892) and Otto Fischer(1861-1917). They 
considered the human body as rigid bodies in form of dynamic links in series. The work of Nicholas Bernstein (1896-1966)
in Moscow introduced the 3D analysis based on cameras. The methods for measuring human movement continued improving 
with the advent of new technologies like electronics and magnetic devices, up until today’s motion capture systems 
based on reflective markers, magnetic or inertial devices. Recently, there has been a huge development in the motion 
capture systems which evolved the motion study to newer dimensions (discussed in the later chapter).

\section{Humanoid robots}

The idea of building machines which look and move like humans has been explored by philosophers and mathematicians
since antiquity. Nowadays, the concepts of such machines are a part of research in robotics. Humanoid robots can 
be thought of as mechanical, actuated devices that can perform human-like manipulation, including locomotion as 
their main skill for displacement. Well before the first modern humanoid robot, one of the biggest steps towards 
this objective was achieved in 1956 with the first commercial robot manipulator, \textit{Unimate, from Unimation}. 
The automotive industry was the first to benefit from these kinds of manipulator robots. Recently, the development 
of humanoid robots for education, research and services has proved the work in multiple ways. 

\begin{figure}[h!]
\centering
\includegraphics[scale=0.6]{images/intro-3.png}\hfill
\caption[Humanoid robots over decades]{Some Bipedal Android robots over decades \cite{evolution} 
    (a) First humanoid robot by Leanardo Da vinci (b) Steam Mam in 1865 (c) Electric man in 1885 (d) 
    ELEKTRO in 1938 (e) BIPER - 4 in 1984 (f) Tron-XM in 1997 (g) H6 Humanoid robot in 2000 (h) Robot 
    JACK in 2000 (i)  GuRoo in 2002 (j) QRIO, Sony in 2003 (k) Partnar Robot, Toyoto in 2004 (l) TwentyOne
    in 2007 (m) REEM-A in 2007 (n) REEM-B in 2008 (o) NAO in 2008.}\hfill
\label{humanoid-robots}
\end{figure}


Current goals of research in humanoid robots include industrial and social applications in day-to-day life. 
A study was conducted by Tanie, \cite{tanie} and is briefed below

\begin{itemize}
\item maintenance tasks of industrial plants,
\item security service for home and offices,
\item human care, teleoperation of construction machines.
\item cooperative work
\end{itemize}

Since many studies have explored this aspect of robot motion and how to make robots more \textit{human-like} and 
\textit{human-aware}. Human Robot Interaction (HRI) is now a challenging research field and studies on the efficacy 
of humanoid robots in human environments are further proceeded.


\section{Motion Imitation}

Given the resemblance between humanoid robots and human beings, it is natural to look for inspiration in human movements
in order to generate motion for the humanoid. The most straightforward way is to have the robot observe what the human 
does and reproduce that behaviour, i.e. perform imitation. After all, even human beings themselves are able to acquire 
skills by imitation and learning.

~


\begin{figure}[h!]
\centering
\includegraphics[scale=0.5]{images/motion-imitation.png}\hfill
\caption[A motion imitation performance by HRP-2 humanoid robot]{A performance  based on motion imitation done by 
HRP-2 humanoid robot}\hfill
\label{marker-based-system}
\end{figure}


If humanoid robots are to interact with human beings, it is imperative that their gestures are human-like since much of
human communication is non-verbal. Programming each aspect of the motion detail by detail in order to make it human-like
is time-consuming and not fit to handle the immense variety and complexity of human behaviours. Thus, imitation comes
as a more natural and intuitive alternative to classical methods, since more information can be transmitted directly.

\subsection*{Challenges in Imitation}

Human motion cannot be directly transferred to the robot without choosing beforehand which affects the ocean or what
is transferring. The most advanced human robots cannot move completely like a human being. The robots are limited 
by differences to human counterparts such as the number of degrees of freedom, link lengths, motor torques, etc. 
Robots which tried to reproduce the whole human body inevitably have never taken dynamic differences from the human 
body they are trying to represent.

~

While performing imitations, the physical differences have to be taken into account when mapping the register home 
movements to the robot morphology. This was addressed by Pollard et al. [5], who limited the captured human motion 
to a range achievable by the robot by locally scaling angles and velocities in order to preserve as much as possible 
local variations in the motion imitated by the Sarcos robot.


\section{Problem Statement}

Recently, almost every humanoid robots are able to walk and balance in flat indoor environments and there are robots 
proved walking on rugged terrains and uneven planes are possible and achievable. A lot of effort is being done to make
them more autonomous by incorporating the perception, planning and action loop. One of the ultimate objective of the 
humanoid robot, as mentioned before is to create the humanoid motion more human-precise. In this sense, robots require 
real-time imitation processed be higher reactivity compensating the unpredictable nature of human motion.

~

In humans, imitation is an advanced behavior whereby an individual observes and replicates the action of another 
human arguably with more accuracy and precision. However in humanoid robots, these kind of motion imitations are possible
up to kinematic level thorugh perception; the robot can be able to imitate the action posture using its predefined 
configurations and controllers. But to imitate the action at dynamic level, humanoid robots are still struggling to 
approximately copy the dynamic parameters applied during the action. Presently, to copy the action at dynamic level, 
feedback data from human during motion or action is mandatory. The motion data from human action is transferred using 
either marker-based or marker-less tracking systems. The main objective of the thesis is to define realtime dynamics 
motoin imitation and validate it experimentatlly using an affordable humanoiod robot. \textit{NAO} from \textit{Aldebaran
Robotics}. The motion capture system used in this scenario is \textit{Xsens MVN} from manufacturer \textit{Xsens}.

\section{Report Overview}

 \chapter{State of the art}
 
\section{Motion Imitation Problem}

\section{Simplified Humanoid Model}

\section{Balance control}

  \chapter{Dynamics Based Whole Body Imitation}
\label{chapter-3}

Whole body motion imitation of a humanoid robot is a challenging problem due to the kinematic and dynamic complexity of the robot. Humanoids are highly
redundant but loses a higher number degrees of freedom compared to human actor. Mapping the actions and poses performed by human actor is kinematically 
achieved \cite{sakka:hal-01054887,mukherjee2015inverse} even though the motion mapped here are slow paced and non-complex actions. Though the high paced
actions are achieved and imitated on computer graphics \cite{brown2013control}, the problem on humanoid robots are completely different domain. To perform face paced actions, 
dynamics parameters of the robot model need to be taken into account for which only a few solutions are proposed \cite{ramosponce, gucci:hal-01895145}. 
As an additional contribution to the problem, this chapter explains a few concepts and methods to improve the imitation pace using acceleration control 
keeping the dynamic balance of the robot.

\section{Dynamic Considerations}
\label{dynamic-consideration}

The dynamic model of the robot states the relation between the generalized torques $\tau$ and acceleration $\ddot{q}$ of the robot given its dynamic 
parameters like Mass and Inertia. Since the humanoids are not attached to the environment, the feet and other humanoid parts like hand (even though 
an assumption is made that only feet are in contact with the environment for this work) impose additional constraints that needed to be taken cared
for keeping the motion intact and balance at all the time. Thus the representation of the robot must include these constraints in addition to the 
balance constraints for which the control strategy must satisfy at all cost. The dynamic model of the humanoid robot (focused towards NAO robot) 
is discussed in this section.

\subsection{Rigid body dynamics}
\label{rigid-body-dynamics}
The Newton-Euler's equations for a rigid body $\mathcal{B}_j$ that are related to linear momentum $p_j$ and angular momentum $\omega_j$ is formulated as follows.

\begin{figure}[h!]
    \centering
    \includegraphics[scale=0.55]{images/rigid-body.png}\hfill
    \caption{Schematic of a rigid body}\hfill
    \label{fig: rigid-body}
\end{figure}

From equations \ref{eq: NE}, the sum of forces and sum of moments at CoM are given by,

\begin{align}
    \begin{split}
        \sum f_j &= \frac{d}{dt}|_{\mathcal{F}_j}p_j 
                = m_j\dot{v_{CoM_j}} \\
        \sum m_j &= \frac{d}{dt}|_{\mathcal{F}_j} (\mathbf{I}_{CoM_j} \omega_j) 
                =\mathbf{I}_{CoM_j}\dot{\omega_j} + \omega_j \times (I_{CoM_j}.\omega_j)
    \end{split}
\end{align}

where $\dot{v_{CoM_j}}$ is the acceleration of the CoM of body $j$; $\dot{\omega_j}$ is the angular acceleration of the body $j$ and 
$\mathbf{I_{CoM_j}}$ is the inertia matrix of body $j$. The NE equations can also be expressed at the origin $\mathcal{O}$ to the body 
$\mathcal{B}_j$ can be expressed as proposed in \cite{khalil2004modeling} as


\begin{align}
    \begin{split}
        \sum f_j &= m_j\dot{v_j} + \dot{\omega_j} \times ms_j + \omega_j \times (\omega_j \times ms_j)\\
        \sum m_j &= \mathbf{I}_{\mathcal{O}}\dot{\omega_j} + \omega_j \times (I_{\mathcal{O}}.\omega_j) + ms_j \times \dot{v_j}
    \end{split}
    \label{eq: NE-rigid-origin}
\end{align}

where $ms_j = m_j.r_{\mathcal{O}CoM_i}$ is the vector of first moment of inertia and $\mathbf{I}_{\mathcal{O}}$ is the inertia matrix at origin $\mathcal{O}$.
Using the screw notation, equation \ref{eq: NE-rigid-origin} can be rewritten as,


\begin{align}
    \sum w_j = \begin{bmatrix}
        \sum f_j \\
        \sum m_j
    \end{bmatrix} &= 
    \begin{bmatrix}
        m_j\mathcal{I}_3 & \hat{ms_j}^T \\
        \hat{ms_j}^T & I_{\mathbf{O}}
    \end{bmatrix}
    \begin{bmatrix}
        \dot{v_j} \\
        \dot{\omega_J}
    \end{bmatrix} +
    \begin{bmatrix}
        \omega_j \times (\omega_j \times (\omega_J \times ms_j)) \\
        \omega_j \times (I_{\mathcal{O}}\omega_j)
    \end{bmatrix} &= M_j\dot{q} + c_j
    \label{eq: ddm}
\end{align}

where $\mathcal{I}_3$ is the identity matrix of size $3$, $M_j$ is the generalized inertial matrix of body $\mathcal{B}_j$ and $c_j$ is the vector of Coriolis and centrifugal effects.
$\sum w_j$ (alias $\varGamma$) is the wrench representation which holds the sum of forces and moments and the equation \ref{eq: ddm} represent the direct dynamic model for a system.
Then the inverse dynamic model can be given by,

\begin{equation}
    \dot{q} = M_j^{-1}(\varGamma - c_j)
\end{equation}

\subsection{Multi body dynamics}

For modelling multi-body dynamics or tree-structured systems in this case, recursive NE algorithm is the most efficient problem than the recursive Lagrangian equations \cite{khalilunified}.

\begin{figure}[h!]
    \centering
    \includegraphics[scale=0.5]{images/multi-rigid-body.png}\hfill
    \caption{Rigid-body representation of tree structured model}\hfill
    \label{fig: tree-rigid-body}
\end{figure}

The recursive algorithm for multi-body system adapted from equation \ref{eq: NE-rigid-origin} is given by \cite{khalilunified}

\begin{align}
    \begin{split}
        \sum f_i &= f_i - \sum_r f_r + m_i\mathbf{g} - f_e \\
        \sum m_i &= m_i - \sum_r (R_km_k + r_k \times f_k) + ms_i \times \mathbf{g} - m_e 
    \end{split}
\end{align}

where $r_k = r_{\mathcal{O}k}$ is the distance vector from origin to force points on body $\mathcal{B}_i$; $ms_i \times \mathbf{g} = r_{\mathcal{O}CoM} \times m_i\mathbf{g}$
$f_r$ and $m_r$ are the reaction forces and moments respectively exerted by the body $\mathcal{B}_r$ on the body $\mathcal{B}_i$ at point $\mathcal{O}_r$;
$f_e$ and $m_e$ represent the forces and moments exerted on the environment by body $\mathcal{B}_i$. These values are assumed to be known.

\subsection{Dynamic Model of NAO robot}

This section derives the dynamic model of the humanoid robot NAO considering the absence of force sensors in the robot feet. 
The mass, CoM and Inertial matrices of the NAO robot v5 H25 is documented on the aldebaran website \cite{aldebaran-masses}. The mass and coordinates of CoM of individual 
links are taken and processed. The CoM given in the documentation is presented relatively to a local coordinate system $\mathcal{R}_j$ attached to the corresponding body $\mathcal{B}_j, i={1, 2, ..., 25}$.
To define the frame system corresponding to the modified DH rule, one has to rotate, translate or do both to compensate the new model for 
inverse dynamics model.

Let $CoM_j^A = [x_A, y_A, z_A]^T$ be the coordinates of the local reference frame $\mathcal{R}_j$ specified in the documentation \cite{aldebaran-masses}. Then the inertial tensor
considering the system of particles of the body about the center of mass with $r=xyz$ as the position vector is defined as,

\begin{equation}
    \label{eq: inertial-tensor}
    J_j^j = \begin{bmatrix}
        \int(y^2 + z^2)dm && -\int xydm && -\int xzdm \\
        -\int xydm && \int (x^2 + z^2)dm && -\int yz dm \\
        -\int xz dm && -\int yz dm && -\int(x^2 + y^2) dm
    \end{bmatrix} 
\end{equation}

\begin{figure}[h!]
    \centering
    \includegraphics[scale=0.35]{images/flowchart-transformation-frames.jpg}\hfill
    \caption{Rotation and Translation of Body reference frame}\hfill
    \label{fig: body-frame-transformation}
\end{figure}

Since the matrix in equation \ref{eq: inertial-tensor} depends on the coordinates of the CoM, when the reference frame changes, the value of inertial tensors changes automatically \cite{karthikmunirathinam}.
To perform rotation from the provided CoM coordinate $CoM_j^A$ represented from $\mathcal{R}_j$ to the coordinate $CoM_j^M$ represented in $\mathcal{F}_j$, the rotation matrix is defined and the rotation matrix
can be computed as,

\begin{equation}
    \label{eq: rotation-frame}
    ^AJ = {^M}A_A{^A}J({^M}A_A)^T
\end{equation}

and the translation can be computed by \cite{karthikmunirathinam},

\begin{align}
    \label{eq: translation-frame}
        {^M}J + m \begin{bmatrix}
            y^2_A + z^2_A && -x_Ay_A && -x_Az_A \\
            -y_Ax_A && x^2_A + z^2_A && -y_Az_A \\
            -z_Ax_A && -z_Ay_A && x^2_A+y^2_A
        \end{bmatrix} = {^A}J + m \begin{bmatrix}
            y^2_M + z^2_M && -x_My_M && -x_Mz_M \\
            -y_Mx_M && x^2_M + z^2_M && -y_Mz_M \\
            -z_Mx_M && -z_My_M && x^2_M+y^2_M
        \end{bmatrix}
\end{align}


Once the frames are transformed to the respective notation, recursive Newton-Euler formalism of dynamic equation from section \ref{rigid-body-dynamicsdra}
can be used to formulate the dynamic model of the equations.

\subsection{Centroidal dynamics of humanoid robot}

\subsection{Zero Moment Point (ZMP)}

The zero moment point (ZMP) of the legged system can be defined as a point where the reaction force at the contact 
of the foot with the ground doesn't produce any horizontal moment i.e., the point where the total of horizontal inertia and
gravity forces equals zero. ZMP is useful in defining the stability criterion of the legged system using the support polygon of 
the foot. A support polygon is a region of perpendicular projection of the balancing end-effectors (legs) that carry the robot's weight.
A ZMP is a projection of the CoM onto the support polygon and it should lie within the region. Any attempt for the ZMP 
outside the support polygon to an extent will result in robot fall. Having these considerations, the ZMP of a biped system
can be given by,

\begin{equation}
    \begin{split}
        P_{ZMP} = P_{CoM} - \ddot{P}_{CoM}(\frac{z_{CoM} - z_{ZMP}}{\mathbf{g} + \ddot{z}_{CoM}})
    \end{split}
\end{equation}

where $\mathbf{g} = -9.81 m/s^2$ is the gravity acceleration. Fixing these points onto a specific position on the floor is 
over-constraining and leaves less free DOF for the other tasks. However, this precise control over the point positioning is advantageous
for choosing safer positions far from the foot’s edges and for avoiding points too far from the ankle which would require too much torque
to hold the whole body \cite{louisepouble}.


\section{Control Approach}

Proposing a control approach for a humanoid robot imitating the human actor differs from standard humanoid control approaches since 
feedback from both human and humanoid robot needs to be taken into account. This section explains the concepts and the approach 
carried out for humanoid control for motion imitation using Stack of Tasks (SoT).

\subsection{Balance Control}

\subsubsection{CoM Retargetting}

To track the Centre of Mass (CoM) of the human actor, an implementation from \cite{penco:hal-01895145} is adapted to track the normalized offset on
human's CoM relative to the support feet. For this case, a projection on 2D plane is considered. The human CoM $P_{CoM}, H$ can be 
expressed using modified Hanavan Model approximation. The normalized offset $\mathbf{o}$ between 0 and 1 can be computed as follows.


\begin{figure}[h!]
    \centering
    \includegraphics[scale=0.4]{images/human-com-track.png}\hfill
    \caption{Representation of offset projection of human actor \cite{penco:hal-01895145}}\hfill
    \label{fig: human-com-offset}
\end{figure}

\begin{equation}
    \mathbf{o} = \frac{(P_{CoM, H}) - P_{LFoot, H})(P_{RFoot, H} - P_{LFoot, H})}{||P_{RFoot, H} - P_{LFoot, H}||^2}
    \label{eq: human-com-offset}
\end{equation}

where $P_{CoM, H}$ represent the position of CoM projection of human actor. on the horizontal plane; $P_{LFoot, H}$ and $P_{RFoot, H}$ are the position of 
the left and right foot of the human actor respectively. The normalized offset $\mathbf{o}$ has an value of 0.5 during double support and has 0 or 1 
during single support. The robot CoM projection then is calculated as,

\begin{equation}
    P_{CoM, R} = P_{LFoot, R} + \mathbf{o}(P_{RFoot, R} - P_{LFoot, R})
    \label{eq: robot-com-1}
\end{equation}

To retarget also changes of the human CoM that are not on the line connecting the two feet, we first measure the maximum backward and forward CoM displacement
 of the human and of the robot over their support polygon (with the origin lying on the feet line), i.e $\delta_{CoM_{back}, H}, \delta_{CoM_{forw}, H}, \delta_{CoM_{back}, R}$ and 
$\delta_{CoM_{forw}, R}$ respectively. Then the retargetting the human CoM displacement $\vartriangle_{CoM, H}$ within the range such that $-\delta_{CoM_{back}, H} \leq 
 \vartriangle_{CoM, H} \leq \delta_{CoM_{forw}, H}$ can be computed as ,

 \begin{equation}
     \mathbf{o'} = \frac{(\vartriangle_{CoM, H} - (-\delta_{CoM_{back}, H}))}{(\delta_{CoM_{forw}, H} - (-\delta_{CoM_{back}, H}))}
 \end{equation}

 for which the robot CoM displacement is

 \begin{equation}
     \vartriangle_{CoM, R} = \mathbf{o'}(\delta_{CoM_{forw}, R} + \delta_{CoM_{back}, R}) - \delta_{CoM_{back}, R}
 \end{equation}

This displacement is then used in the orthogonal direction of the line connecting the two feet of the robot.

\subsubsection{Floating base Control}

To control the height of the floating base of the robot,, the pelvis point of the human actor is considered. The deviation of the pelvis point of the human is given by,

\begin{equation}
    \vartriangle_{base_{t, H}} = base_{t,H} - base_{0, H} 
\end{equation}

where $t$ is the timestep such that $t \geq 0$. Then the correction for robot base is given by,

\begin{equation}
    \vartriangle_{base_{t, H}} = \frac{h_{base, R}}{h_{base, H}} \vartriangle_{base_{t, R}}
\end{equation}


where $\alpha =  \frac{h_{base, R}}{h_{base, H}} $ is the ratio of height of the floating base of the robot and of the pelvis of the human,
at N-pose. Then the height of the robot base at each timestep can be calculated by,


\begin{equation}
    base_{t, R} = base_{0,R} + \vartriangle_{base_{t, R}}
\end{equation}


The change of orientation of the floating base is also calculated in a similar way, by computing the roll, pitch and yaw from the quaternion information given by the motion capture system.

\subsubsection{ZMP Retargetting}

During whole body teleoperation of humanoid robots, disastrous crashes may occur if the desired CoM trajectories recorded from the human do not ensure the balance of the controlled robot when retargeted.

To this scope, we propose a QP-based “preprocessor” that adjusts in real-time the desired commanded CoM to satisfy constraints that represent a condition for dynamic balance. In order to achieve a stable CoM trajectory we employ the linear inverted pendulum model (LIPM) in combination with the Zero Moment Point (ZMP) criterion.
The ZMP is represented with a point on the ground plane where the tipping moments, generated by the gravity and the inertial forces, are equal to zero. A humanoid robot keeps its balance if the ZMP is contained inside the support polygon of the robot.

Through the LIPM model it is possible to establish a simple relation between the ZMP and the CoM dynamics:

\begin{equation}
    \ddot{p}_{CoM} = \frac{g}{h}(p_{CoM} - p_{ZMP})
    \label{eq: zmp-control}
\end{equation}

\subsection{Posture Control}
\label{sec: posture-control}

This section provides a method for the simplification of humanoid robot and the human actor as a composition of inverted pendulum models. These simplified multi-double inverted pendulum models are used for posture control
during the motion retargetting.

\subsubsection{Multi-Double Inverted Pendulum Model (M-DIP)}
The full humanoid robot may be simplified using a combination of different double inverted pendulums as represented in figure \ref{fig: human-mdip}. The current position of the individual pendulums is represented as $\mathbb{P}_i$ and $\mathbf{P}_i$ for
human actor and humanoid respectively where i represent the left hand ($lh$), right hand ($rh$), right leg ($rl$) and left leg ($ll$). Then the state of the human actor $\mathbb{P}$ and humanoid robot $\mathbf{P}$ is represented as, 


\begin{align*}
    \label{eq: pendulum-state}
        \mathbb{P} = \begin{bmatrix} \mathbb{P}_{rh} && \mathbb{P}_{rl} && \mathbb{P}_{lh} && \mathbb{P}_{ll} \end{bmatrix}^T \qquad
        \mathbf{P} = \begin{bmatrix} \mathbf{P}_{rh} && \mathbf{P}_{rl} && \mathbf{P}_{lh} && \mathbf{P}_{ll} \end{bmatrix}^T
\end{align*}


\begin{figure}[h!]
    \centering
    \includegraphics[scale=0.25]{images/flowchart-xsens-pendulum.jpg}\hfill
    \includegraphics[scale=0.2]{images/flowchart-NAO-pendulum.jpg}\hfill
    \caption{Human actor and NAO robot as M-DIP model }\hfill
    \label{fig: human-mdip}
\end{figure}

The individual pendulums states are comprised of roll and pitch angles in the robot frame and can be used the formulate the equation of motions from equation \ref{eq: inverse-dynamic-model} \cite{pierro2012stabilizer}.
The mass, velocity and acceleration contribution of the M-DIP as composition can be computed as,

Mass:
\begin{equation}
    \label{eq: MDIP-mass}
    \mathbb{M} = m_{rl} + m_{ll} + m_{rh} + m_{rh}
\end{equation}

Velocity:
\begin{equation}
    \label{eq: MDIP-velocity}
    \mathbb{\dot{X}}_i = m_il_{sl}l_i\begin{bmatrix}
        0 && -\sin(\gamma_{sl} - \gamma_i)\dot{\gamma}_i \\
        \sin(\alpha_{sl} - \alpha_i)\dot{\alpha}_i && 0
    \end{bmatrix} \qquad \mathit{,i = rh, rl, lh, ll}
\end{equation}


Acceleration:
\begin{equation}
    \label{eq: MDIP-acceleration}
    \ddot{\mathbb{X}}_i = m_il_{sl}l_i\begin{bmatrix}
        0 && -\cos(\gamma_{sl} - \gamma_i) \\
        \cos(\alpha_{sl} - \alpha_i) && 0
    \end{bmatrix} \qquad \mathit{,i = rh, rl, lh, ll}
\end{equation}

where $i=sl$ represent the pendulum in contact with the environment (in this case, the legs); $\alpha_i$ and $\gamma_i$
represent the roll and pitch angles of the respective pendulums $i$ respectively. Given the velocity and acceleration contribution of the M-DIP to the composition model, the position of ZMP can also be retrieved
as, 

\begin{equation}
    \label{eq: MDIP-zmp}
    p_{ZMP, H} = \frac{1}{\mathbf{g}\mathbb{M}}\{\mathbb{\ddot{X}}\mathbb{\ddot{P}} + \mathbb{\dot{X}}\mathbb{\dot{P}} + \mathbb{M}l_{sl}\mathbf{g}\begin{bmatrix}
        \sin(\gamma_{sl}) \\
        -\sin(\gamma_{sl})
    \end{bmatrix}_{H}\}
\end{equation}

The equation \ref{eq: MDIP-zmp} can also be rewritten for calculating the postion of robot ZMP as,

\begin{equation}
    \label{eq: MDIP-robot-zmp}
    p_{ZMP, R} = \frac{1}{\mathbf{g}\mathbf{M}}\{\mathbf{\ddot{X}}\mathbf{\ddot{P}} + \mathbf{\dot{X}}\mathbf{\dot{P}} + \mathbf{M}l_{sl}\mathbf{g}\begin{bmatrix}
        \sin(\gamma_{sl}) \\
        -\sin(\gamma_{sl})
    \end{bmatrix}_{R}\}
\end{equation}

where,
\\
$\mathbb{P}_{i} = [-\arctan(\frac{-y_i}{z_i}) \quad \arctan(\frac{-y_i}{z_i})]^T$ is the position state of the respective pendulum $i$; \\
$\mathbb{\dot{X}} = [\mathbb{\dot{X}}_i, \mathit{i = rh, rl, lh, ll}]^T$ is the velocity contribution of the pendulum model of human; \\
$\mathbf{\dot{X}} = [\mathbf{\dot{X}}_i, \mathit{i = rh, rl, lh, ll}]^T$ is the velocity controibution of the pendulum model of robot; \\
$\mathbb{\ddot{X}} = [\mathbb{\ddot{X}}_i, \mathit{i = rh, rl, lh, ll}]^T$ is the acceleration contribution of the pendulum of human; \\
$\mathbf{\ddot{X}} = [\mathbf{\ddot{X}}_i, \mathit{i = rh, rl, lh, ll}]^T$ is the acceleration contribution of the pendulum of robot.

\subsubsection{Acceleration Control}

The human actor and the robot are simplified to the composition of M-DIP model and the posture control can be implemented directly using the acceleration feedback between the 
two subjects. The acceleration control for the motion-retargetting can be implemented as a weighted-constraint based problem \cite{van1985method},

\begin{equation}
    \label{eq: MDIP-control}
    \lim_{a \Rightarrow \infty} (HJ)^+ H\dot{e}^* = argmin (||H_iJ_i\dot{q} - H_i\dot{e}^*_i||^2), \quad \mathnormal{s.t. \quad J_i\dot{q} = \dot{e}_i^*}
\end{equation}


where 
\section{Task Specification Approach}

\subsection{Joint trajectory tracking}
\subsection{Balance Control}
\subsection{Posture Control}
\subsection{Joint Limits Avoidance}

  \chapter{Implementation}
\label{chapter-6}

This chapter primarily discusses the steps and modifications that has been carried out to tune the software. The imitation 
is performed initially with upper body and then to lower body eventually testing the complex motions considering the NAO robot's 
structure. The results of the implementation will be discussed in the coming sections later in this chapter.


\section{Methodology}

The implementation began by testing the robot's kinematics already implemented in order to obtain relation between it's joints  and task spaces.
Once the suitable implementation has been validated, the captured human data is scaled down to the robot's size and it's desired parameters $\Omega$
are retrieved for HQP.

\begin{figure}[h!]
    \centering
    \includegraphics[scale=0.5]{images/flowchart-block-diagram.png}\hfill
    \caption{Block Diagram Representation of Imitation Program}\hfill
    \label{fig: block-diagram}
\end{figure}

Figure \ref{fig: block-diagram} represents the flow of data between individual blocks discussed in chapter \ref{chapter-3} and 
chapter \ref{chapter-4}. It is notable that two computers are used for this implementation to extract convenient results.

\section{Robot Model}

During the course of this research, both real robot and virtual models are used. The real robot used for validation is
a NAO v5 H25 Humanoid robot and the virtual model is a CopelliaSim model with approximately equivalent inertial modelling.
Like previous implementations on kinematic modelling \cite{louisepouble}, the hand operations (open/close) will be ignored therby
controlling 23 DoFs in total (vector $q$). Figure \ref{fig: nao-robot-joint} represents the arrangement of DoFs that will be used in this research.

\begin{equation*}
    q = \begin{bmatrix}
        q_1 && q_2 && ... & q_{23}
    \end{bmatrix}^T
\end{equation*}
\begin{figure}[h!]
    \centering
    \includegraphics[scale=0.55]{images/nao-robot-joint.png}\hfill
    \caption{NAO Robot's DoF Arrangement \cite{louisepouble}}\hfill
    \label{fig: nao-robot-joint}
\end{figure}

\subsection{Kinematic Modelling}

The robot's geometric model was calculated using the nominal approach with \textit{Modified Denavit-Hartenberg} (MDH) parameters. Since the 
robot is a tree structure, some links have more than one child links. In addition, it might be convenient to include 
frames in the model which are not necessarily defined according to the previous rules. Then, the kinematic model can be represented as,

\begin{equation}
    \label{eq: robot-dgm-1}
    \begin{bmatrix}
        \dot{x}_{lhand} \\ \dot{x}_{rhand} \\ \dot{x}_{lleg} 
    \end{bmatrix} = J \begin{bmatrix}
        \dot{q}_{lhand} \\ \dot{q}_{rhand} \\ \dot{q}_{lleg}
    \end{bmatrix}
\end{equation}

Equation \ref{eq: robot-dgm-1} can be simply represented as $\dot{X} = J\dot{q}$ in the next sections where $J$ is the robot Jacobian and $x_i$ is a $3 \times 1$ vector without
considering the end-effector oriendation. The above equation has a vector size of $(6 + 6 +5) \times 1$ considering only three kinematic chains for the sake of redundancy. 
In other words, the right foot will always be in contact with the floor. 
This work was implemented in MATLAB and was exported to a C++ library "Jacobian" (available in \cite{github}). 

\subsection{Masses and CoM}

The robot's dynamic parameters like masses and Centre of Masses can be obtained from the 
documentation directly \cite{aldebaran-masses}. The individual masses and CoM for each link 
is calculated and documented in the website. To calculate the single point mass and the projection of 
CoM onto xy plane is,

\begin{align}
    \label{eq: robot-mass-com}
    M &= \Sigma_{i = 1}^n m_i \\
    P_{CoM} &= \frac{1}{M}\Sigma_{i = 1}^n m_ic_i
\end{align}

where $m_i$ and $c_i$ represent the individual mass and CoM of link $i$ obtained from documentation respectively; n is the number of moving masses 
in the robot body. The jacobian of CoM can then be formulated with respect to the robot frame (frame located at right foot) is given by,

\begin{align}
    \label{eq: robot-mass-com-2}
    J_{CoM} &= \frac{\partial P_{CoM}}{\partial q} \quad \mathit{or} \\
    \dot{P}_{CoM} &= J_{CoM} \dot{q}
\end{align}

The size of $J_{CoM}$ is $2 \times 23$ (reflection of CoM onto 2D plane (x, y)). $P_{CoM}$ is a 2-dimensional vector since the projection on xy plane.

\section{Human Motion}

The sensor data acquired from Xsens suit is high-quality less-noise data with the ability to record wide range of motions.
The data acquired greatly differ in cartesian sizing compared to the size of the NAO robot. Usually in motion imitation, the robot 
must reach the same point in task space as the human actor, but due to greater difference in size, the task space is treated as relative instead of absolute. 
To achieve this criteria, a scaling function need to be implemented before
feeding the Xsens data to the robot control.

\subsection{Scaling Function}

In a previous work \cite{scaling-human-nao}, the use of a scaling factor for each limb based on the ratio between the lengths of the kinematic chain of actor and 
imitator was proposed. This factor was then multiplied by the captured human data to obtain the desired positions for the robot. Since each limb 
was a simple serial chain, a proportional scaling factor could be applied satisfactorily.

However, in the present work, the balance is being taken into account and thus the robot has to be modelled as a whole, which gives rise to s 
tree-structure. In this case, keeping the direction of the captured human data means that the robot’s end-effectors will fall on the line which 
connects the tracked points to the origin on the right foot.

\begin{figure}[h!]
    \centering
    \includegraphics[scale=0.7]{images/proportional-scaling.png}\hfill
    \caption{Proportional scaling of human motion \cite{louisepouble}}\hfill
    \label{fig: proportional scaling}
\end{figure}

A new scaling method is proposed in \cite{sakka:hal-01054887} which takes into account not only the
lengths of each of the segments but also each of their directions. This way, the captured
human motion is scaled segment by segment to the lengths of the robot while keeping the 
direction of that segment. The iterative process starts from the right ankle and moves upwards towards each of the end-effectors.

In summary considering a link $l$, the scaling can be processed as,

\begin{equation}
    P_l^{robot} = \frac{P_l^{human} - P_a^{human}}{||P_l^{human} - P_a^{human}||}l_l^{robot} + P_a^{robot}
\end{equation}

~

where $P_a$ refers to the antecedent position to the link $l$. Since the robot doesn’t have a spine, the distances connecting shoulders and hips 
should remain constant. The shoulders cannot be scaled directly from the hips because of the different torso proportions between the human and the 
robot. It was chosen to scale a segment which goes from the point between the hips to the point between the shoulders, preserving symmetry.
 
 
 \chapter*{Conclusion}
 \addcontentsline{toc}{chapter}{Conclusion}
 
 
 
 
 
 % switch to A-B-C chaptering
 \appendix	
 
 
 \addcontentsline{toc}{chapter}{Bibliography}
 
 \bibliography{biblio}
 
 
 
 
\end{document}
