\documentclass{report}


\begin{document}

 \thispagestyle{empty}

\def\lskip{\vspace{0.5cm}}


\begin{tabular}{p{7cm}p{8cm}}
ÉCOLE CENTRALE DE NANTES
&
% EMARO students only
% \raggedleft FIRST YEAR INSTITUTION	
\end{tabular}

\vspace{2cm}

% CORO-IMARO students
\begin{center} \large\sc MASTER CORO-IMARO\\ \normalsize{``CONTROL and ROBOTICS''} \end{center}

% EMARO students
%\begin{center} \large\sc MASTER ERASMUS MUNDUS \\ \normalsize{EMARO+ ``European Master in Advanced Robotics''} \end{center}


\begin{center}
	2020 / 2021\\
	\lskip
	Master Thesis - Midterm Report % or bibliography report
	\lskip
	
	Presented by \lskip 
	
	Selvakumar Hastham Sathiya Satchi Sadanandam \lskip
	
	On 17 May 2021 \lskip\lskip
	
	{\Large \textbf{Realtime human motion imitation by humanoid robot with balance constraint}}
	
	\vfill

% Jury \lskip
		
	\end{center}
	


\begin{tabular}{p{3cm}p{5cm}p{7cm} }
 % President: & Name & Position (Institution) \\ & & \\     % for final defense only (not bibliography)
%  Evaluators: & Dr. Olivier Kermognant & Assistant Professor (LS2N, ECN) \\
% 	      & Name & Position (Institution) \\ 
% 	      & Name & Position (Institution) \\ & & \\  & & \\ 
  Supervisor(s):  & Dr. Sophie Sakka & Assistant Professor (LS2N, ECN) \\
% EMARO students only
%(EMARO)  & Co-supervisor from M1 & Position, M1 institution 
\end{tabular}

\lskip

\begin{flushleft}
 Laboratory: Laboratoire des Sciences du Numérique de Nantes LS2N
\end{flushleft}

\newpage
\thispagestyle{empty}
\null
\newpage
\addtocounter{page}{-1}
\pagestyle{fancy}
  
 
  \section*{Abstract}
   
  Humanoid robots are made as mirror to the humans. Consequently, the humanoid motion is also expected to be real as 
  human and it is natural to use human motion as an input to generate humanoid motion. This process is called motion 
  imitation and there are several challenges posed due to kinematics and dynamics of the robot over the past decades. 
  Although the work on the kinematic challenges is actively improving and notably better than dynamics, it allows the 
  robot only to move and imitate slow actions. For the fast paced motions, \textit{momentum} gets build up and needs 
  dynamics to be taken into account. Due to the differences in redundancy between humanoid robots and humans, real-time 
  imitation in humanoid robots while keeping balance and support changes is still an unsolved problem that need to be 
  addressed.

  ~
  
  During this research, imitation based on dynamics with task optimisation will be implemented on humanoid robot 
  \textit{NAO}, from Aldebaran Robotics. The imitation will be carried out online using marker-based motion capture
   system from \textit{Xsens}, specifically \textit{Xsens MVN} system.
  
  ~
  
  Initially the captured motion will be preprocessed for its representation in operational space and then will be 
  scaled to the robot's dimensions. From this scaled motion, the desired poses are taken and joint spaces in 
  \textit{NAO} and human actor will be mapped using the scaling function directly. The scaling constraints and the 
  joint angle limits are taken into account for this process. Body Segment Parameter(BSP) Estimation is carried out 
  on the scaled frames and the approximated \textit{mass and Centre of Mass (CoM)} are calculated based on the 
  principles of Modified Hanavan Model. The balance equations and the CoM trajectory are taken as the optimisation 
  problem for the motion imitation. Finally the work will be presented as non-differential optimisation problem 
  considering dynamics equations of motion and will be validated real-time on the humanoid robot.
 
 
 \newpage
 
 \section*{Acknowledgements}
 
 I would first like to thank my thesis advisor Dr Sophie Sakka for her motivation and opportunity for the thesis topic.
  I would like forward my thanks to Prof. Olivier Kermorgant for his continuous motivation and help in informing 
  students about thesis positions without which I would not have found the position. 

~

I would also like to thank Prof. Ina Taralova for her indispensable advice regarding the rules and practices involved 
in the organisation of this report , which I have tried to implement to the best of my abilities.

~

Finally, I must express my very profound gratitude to my parents and my friends for providing me with unfailing 
support and continuous encouragement throughout my years of study and through the process of researching and writing 
this report. This accomplishment would not have been possible without them.
 \newpage
 
 
\section*{Notations}
    \begin{tabular}{p{3cm}p{10cm}}
    CoM & Centre of Mass
    \end{tabular}

\newpage

\section*{Abbreviations}

\begin{tabular}{p{3cm}p{10cm}}
CoM & Centre of Mass \\
ZMP & Zero Moment Point \\

\end{tabular}

 \newpage
 
 \tableofcontents
 
 \listoffigures
 
\listoftables
 

 
 
 \chapter*{Introduction}
 \addcontentsline{toc}{chapter}{Introduction}	 % non-numbered chapters do not appear in table of contents by default
 
 
 \chapter{State of the art}
 
 \section{First topic}
 
 \section{Second topic}
 
 \chapter{Actual work}
  
 
 When dealing with rectangled triangles (see Figure \ref{triangle}) I sometimes used this theorem from \cite{pythm001}:
 \begin{equation}\label{theo}
  a^2 + b^2 = c^2
 \end{equation}The demonstration is in Appendix \ref{sec:prooftheorem}.
 

 
 \chapter{Experiments}
 
 When trying to draw a rectangled triangle, my program comes up with Figure \ref{triangle2} that is neither rectangled nor a triangle.
 

Unless there is a bug in my program, which is unlikely, this research indicates that the whole theory on triangles having 3 sides has been wrong for years, maybe decades.
 
 
 \chapter*{Conclusion}
 \addcontentsline{toc}{chapter}{Conclusion}
 
 
 
 
 
 % switch to A-B-C chaptering
 \appendix	
 
 \chapter{Proof of theorem \ref{theo}}
 \label{sec:prooftheorem}
 
 
 \begin{proof}
\eqref{theo} was already demonstrated in \cite{euclides300}.
\end{proof}
 
 \addcontentsline{toc}{chapter}{Bibliography}
 
 \bibliography{biblio}
 
 
 
 
\end{document}
