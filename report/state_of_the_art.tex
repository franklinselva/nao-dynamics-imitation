\chapter{State of the Art}
\label{chapter-2}

The research and interest in humanoid robotics has greatly increased over the last few years, but inspite of the current
abundance of the humanoid robots, their utility is still very limited. One of the most important trials, \textit{DARPA 
Robotics Challenge (DRC)} in 2013 listed a pack of capabilities and robustness that humanoid robots lacked when performing 
different tasks. Each task of this challenge should last only up to 30 minutes. These tasks will take less than a minute 
for a human to complete which  explains the powerlessness of the robots. This chapter briefly explains different control 
strategies that has been carried to keep balance and to handle robot dynamics in humanoid robots.

\section{Approaches in Humanoid Robot Control}

Different methods has been used to make a robot move depending on the application, the complexity of the task, and even
the specific nature of the robot. The main control methods used for the humanoid robots are as follows: Motion planning,
Kinematic Approach, Dynamic Approach and Optimal Control. Each of the methods and its recent development are discussed
below.

\subsection{Motion Planning}

Motion Planning as the name suggests a method where a robot automatically finds its desired or goal state from its initial
configuration. For instance, consider a hand moving from it's current pose to another pose, motion planning allows to move to
goal pose considering the presence of obstacles and consumption of time and energy. Currently, there are many applications in 
industrial robots and mobile robots where the robot motion is planned within both structured and non-structured environment. 

 

Even though, motion planning and its researches are progressing majorly within past decades, the recent approaches are combined
with artificial intelligence, advantages of computer technology and mathematics. The main approaches can be found in books such as 
\cite{Latombe,LaValle2006PlanningA}. An desirable concept in motion planning is \textit{Configuration Space (CS)}, which is the set
of all possible configurations for a robot to attain. For a robot with \textit{n} independent degrees of freedom, \textit{CS} is an 
\textit{n}-dimensional manifold $\mathbb{M}$ that contains all the desired configurations $q \in \mathbb{M}$ of the robot. The importance
of \textit{CS} is that changes the problem of moving a body in \textit{SE(3)} to moving a point in \textit{CS}. The summary of this
section is sourced from the literature work \cite{ramosponce}. Then there exists,

\begin{itemize}
    \item $\mathit{CS_{obs}}$ is the \textit{Obstacle Configuration Space} formed to generate self-collision or obstacle collision free set of
     configurations such that $\mathit{CS_{obs}} \in \mathit{CS}$.
     \item $\mathit{CS_{free}}$ is the \textit{Free Configuration Space} which holds the set of configurations for a freely roaming robot such that 
     $(\mathit{CS_{free}} \bigcup \mathit{CS_{obs}}) \cap \mathit{CS} $.
\end{itemize}

Using these configurations, the problem of motion planning can be stated as finding the continuous path $p(t)$ through the desirable configurations from initial
state $q(0)$ to goal state $q(f)$ avoiding collisions, that is $p: [0, 1]  \rightarrow \mathit{CS_{free}}$ where $t$ defines time parameterization.


\subsubsection{Generic methods}

The solution to motion planning problem can be processed through classical approaches using deterministic, sampling-based or path optimization algorithms. 
Each of the types are briefed below.

\begin{figure}[h!]
    \centering
    \includegraphics[scale=0.3]{images/motion-planning.png}\hfill
    \caption{Sampling based motion planning of NASA Valkyrie \cite{Vijayakumar}}\hfill
    \label{motion-planning}
\end{figure}

\begin{enumerate}
    \item \textit{Deterministic algorithms -} The deterministic algorithms are developed such that it computes the valid path everytime knowing almost
    all the variables of the environment. Methods such as \textit{cellular decomposition, Voronoi diagrams, visibility graphs, potential fields and Canny's algorithms}
    rely on
    mathematical construction of the environment with the obstacles and provide $\mathit{CS_{obs}}$. Although these algorithms are complete, the computation
    of high-dimensional space is expensive and the environments are always not deterministic.

    \item \textit{Sampling based algorithms -} These algorithms mostly approximate the connectivity of $\mathit{CS_{free}}$ through random sampling configurations
    from $\mathit{CS}$ and rejecting the configurations using boolean collision detection techniques. The main examples are  \textit{Probabilistic Maps and
    Rapidly-exploring Random Trees(RRT)} in combination with Voronoi diagrams promotes the obstacle avoidance configurations for a robot. The main advantage of
    sampling based algorithms are to handle the higher dimensional configuration space recovering a higher degree of completeness.

    \item \textit{Path optimization algorithms -} These algorithms provide optimization in terms of path planning and trajectory planning starting from a valid
    initial state to its goal position with the desired configurations. \textit{Greedy optimization} tries to directly connect the start configuration to 
    its goal state that generates a collision free shortest path by discretizing the path into \textit{n} closest goal configuration relative to previous 
    configuration.
\end{enumerate}

\subsubsection{Motion Planning in Humanoid robots}

Classical motion planning techniques determine collision free trajectories considering only the geometric model of the robot. However the control of
polyarticulated system needs the synthesis of robot models that describe the effect of joint variations on the whole robot configuration. In both the case,
for instance considering an arm moving to it's goal position, the robot tends to make it more unusual, inefficient and unnatural movements. To overcome these 
problems, geometric models are replaced with kinematic models, dynamic models or optimal control and trajectories are generated. Additional constraints like
multiple contacts and dynamic balance of the system are considered. For instance, motion primitives that have been predefined by a human expert based on prior
knowledge can be used to guide the planner \cite{zhang2014motion}. 

 

In case of humanoid walking, the planner can be generated using deterministic approaches with dynamic alterations of the foot transition model considering
the smooth transition of the trajectories for posture transitions. For these cases, sampling-based algorithms are considered to improve the degree of completeness.
Either way, the higher dimensional configuration space is handled so that the problem is solved successively. An example is presented in \cite{yoshida2008planning} where a 36 degree of freedom
robot is reduced to a 3 degrees of freedom bounding box and a PRM is applied for the path planning problem of the box. Another example is to present the
constraints in the form of sub-manifolds of \textit{CS} where a union of separate manifolds like contact limb position and static balance constraints can be 
used to plan the configuration space. In such cases, static balance control in humanoid robots and other legged robots can be obtained \cite{hauser2010multi}.

\subsection{Kinematics Approach}

Generally, \textit{kinematics} is defined as a branch of science which deals with the study of the position, velocity and acceleration of a mechanical system without
considering forces and the dynamic properties of the system (such as mass or inertia) that generate the motion. In humanoid and manipulator robots, the system is 
represented as rigid bodies composed of actuators and sensors. In contrast with the manipulators, the humanoid robots are not fixed to any environment and are highly mobile.
This makes the humanoids (or humanoid robots) more redundant than the manipulators. This section briefs the state of the art and concepts of kinematic approach used in
humanoid control.

\subsubsection{Basic Concepts}

The \textit{joint space} is also called as \textit{configuration space}, of a robot with $n$ degrees of freedom (DoF) is a \textit{n}-dimensional manifold $\mathit{Q}$ containing 
all the possible joint values for a joint $q$ can take. For humanoid robots, this space can be generalized to the operational points \cite{khatib1987unified}, which can represent any part of the body
that may be of interest. In robotics, there exists four subdomains of kinematics namely,  \textit{\textit{(i)} forward kinematics (or direct geometry), \textit{(ii)} inverse kinematics
(or inverse geometry), \textit{(iii)} forward differential kinematics (or simply forward kinematics), and \textit{(iv)} inverse differential kinematics (or simply inverse kinematics).}

\begin{itemize}
    \item \textbf{Direct Geometric Model:} For a robot with $n$ DoF in a $n$-dimensional joint space such that $q \in Q$, there exist a pose $x \in \mathit{SE(3)}$ represented as
    $$x = f(q)$$ described by a map $f:Q \rightarrow \mathit{SE(3)}$.
    
    \item \textbf{Inverse Geometric Model:} For a robot with $n$ DoF in a $n$-dimensional joint space with $q \in Q, x \in \mathit{SE(3)}$, the joint space can be represented from 
    a given pose for a certain operational point as $$q = f^{-1}(x)$$ described by a map $f: \mathit{SE(3) \rightarrow Q}$. But there is a possibility for non-unique solution or 
    non-existing solution (known as singularity).

    \item \textbf{Forward Kinematic Model:} For a robot with $n$ DoF in a $n$-dimensional joint space such that $q \in Q, x \in \mathit{SE(3)}$, the operational twist $\xi \in
     \mathit{SE(3)}$ due to the joint variation $\dot{q}$ is described as $$\xi = J(q)\dot{q}$$ Here, $J$ is the basic Jacobian such that $J : T_q(Q)$ where $T_q(Q)$ is the tangent 
     space of the $Q$ space. Instead, the Jacobian $J$ can be formulated analytically using the pose variation $\dot{x}$ and joint variation $\dot{q}$ as $$\dot{x} = \frac{\partial x}{\partial 
     q}\dot{q} \quad \mathnormal{or} \quad \dot{x} = J\dot{q}$$ then $J$ is the task Jacobian.

     \item \textbf{Inverse Kinematic Model:} For a robot with $n$ DoF in a $n$-dimensional joint space such that $q \in Q, x \in \mathit{SE(3)}$, finding the joint variations $\dot{q}$
     that produce a pose variation $\dot{x}$ of the end effector as $$\dot{q} = J^{-1} \dot{x}$$ and it can be solved iteratively.
\end{itemize}

\subsubsection{Kinematic Control of Redundant robots}

In general, the kinematic control involves a reference planner or trajectories $q_{ref}(t)$ in $\mathit{SE(3)}$ for which the robot joint trajectory $q(t)$ is evaluated against attaining
the objective. Consider a task $i$ in a robot with $n$ DoF, then the Jacobian is of size $m \times n$ and the robot is redundant when $(n > m)$ \cite{nakamura1990advanced}. The joint variations $\dot{q}$ can be represented
if

\begin{itemize}
    \item $(n > m)$. For task $i$, there exists a degree of redundancy $n-m$ for which the joint variation based on lease square method can be given by $$\dot{q} = J_i^+\dot{x_i} + (I_n - J_i^+J_i)z_i$$
    where $J_i^+ = J_i^T(J_iJ_i^T)^{-1}$ is the pseudo inverse of $J_i$, $I_n$ is the identity matrix of size $n$ and $z_i$ is an $n$-dimensional arbitrary vector. The first term in the above equation
    is to minimize the norm solution while the second term is to find all possible solutions. 

    \item $(n = m)$. The degree of redundancy is $0$ and the joint variation can be defined as $$\dot{q} = J^{-1} \dot{x}$$ Note that the Jacobian here is a non-singular matrix.
    \item $(n < m)$. There doesn't exist any redundancy and no solution is found in this case.
\end{itemize}

For multiple tasks, the probability of finding a suitable solution decreases as the preceding tasks influence the current task, in other words, consider two tasks 1 and 2 for a redundant robot. There 
exist a solution for two tasks $x_1 = f_1(q)$ and $x_2 = f_2(q)$ with task priority for $x_1$ and $x_2$ respectively. First the variation $q$ that solves the task according to the priority is 
determined from the differential equations by,

\begin{equation}
    \delta x_1 = J_1\delta q 
    \label{task1}
\end{equation}
\begin{equation}
    \delta x_2 = J_2\delta q 
    \label{task2}
\end{equation}

Then the joint variation $\delta q$ for task $1$ has infinitely many solutions and is given by

\begin{equation}
        \dot{q} = J_i^+\dot{x_i} + (I_n - J_i^+J_i)z_i 
        \label{jvariation}
\end{equation}

Solving \ref{jvariation} and \ref{task2} for $z_1$ arbitrary vector,

\begin{equation}
    z_1 = \hat{J_2^+}(\delta x_2 - J_2J_1^+ \delta x_1) + (I_n - \hat{J_2^+}\hat{J_2})z_2
\end{equation}

where $\hat{J_2} = J_2(I_n + J_1^+J_1)$ and $_2$ is an $n$-dimensional arbitrary vector. Then the solution of joint variation for the two priority tasks can be given by 

\begin{equation}
    \delta q = J^+_1 \delta x_1 + \hat{J_2^+}(\delta x_2 - J_2J_1^+ \delta x_1) + (I_n - J^+_1J_1)(I_n - \hat{J_2^+}\hat{J_2})z_2
\end{equation}

\subsubsection{Generalized form of Task Priority}



\subsubsection{Kinematic Control in humanoid robots}

Humanoid robots as a rigid body representation, from a kinematic point of view present a tree-like structure that includes multiple connected chains, and a high number of DoF. This results in 
higher degree of redundancy with respect to most tasks, in which case the robot is said to be under-constrained. Therefore, methods developed for generic redundant robots are usually applied in humanoid robotics
with some adaptations or additions. Due to the complexity of the kinematic configuration, closed-form solutions for the IK problem are usually very complex, but recently some classical methods 
have been used and modified, to obtain closed equations for specific humanoid robots which are treated as a composition of several kinematic chains. However, methods based on 
instantaneous IK, which compute an increment in q, are usually preferred. This linearization of the problem offers an infinite number of feasible solutions for humanoid robots: there exist different
joint updates that achieve the same task. This leads to the possibility of performing different tasks at the same time, and the IK control must be capable of properly handling them.

 

Methods based on task-prioritization solve these problems and have become the preferred techniques in IK control. They can even be considered as the current “state of the art” in humanoid 
robotics control due to their relatively low computational cost, their straightforward implementation, and the maturity of the approach. Several works with different humanoid platforms use this 
methodology to solve the redundant IK problem at the velocity level using only equality constraints \cite{gienger2005task,yoshida2006task,mansard2007task}. For imposing inequality constraints 
to the control framework at any hierarchical level, a sequence of optimal resolutions for each priority level has been proposed in \cite{kanoun2009prioritizing}, a more efficient computation based on orthogonal 
decompositions can be found in \cite{escande2013planning} and a smooth interchange between priority of consecutive prioritized tasks is introduced in \cite{jarquin2013real}.

\subsection{Dynamics Approach}

Dynamics is the study of the relation between the robot motion and the generalized forces that act on the robot generating that motion. This relation considers parameters such as lengths, 
masses and inertia of the elements composing the robot.



\subsubsection{Basic Concepts}

In the dynamic model, the motion is represented through joint variables acceleration $\ddot{q}$, or operational points acceleration $\ddot{x}$. For rotational joints, the generalized forces are equivalent to 
the joint torques, and for prismatic joints they are the joint forces. There are two main problems in dynamics:

\begin{itemize}
    \item \textbf{Forward Dynamics:} It expresses the motion of the robot as a function of the generalized forces applied to it.
    \item \textbf{Inverse Dynamics:} It expresses the generalized forces acting on a robot as a function of the robot motion.
\end{itemize}

There exist two main formulations to compute the robot dynamic model: the Lagrange approach, and the Newton-Euler approach. The Lagrange approach's main advantage is the clear separation 
of each component of the model; but in general, it is computationally expensive.
The Newton-Euler approach does not provide a clear separation of the terms but due to its recursivity a lower computation time can be obtained. Thus, it is 
the preferred implementation for computer calculations. The most used algorithms for this approach can be found in \cite{featherstone2000robot,featherstone2014rigid}. The Lagrange method and 
Newton Euler's method are detailed below %in the chapter \ref{chapter-3} under section \ref{dynamic-consideration}.

\begin{enumerate}
    \item \textbf{Lagrange Formulation}
    
    The Lagrange formulation descibes the behavious of a dynamic system in terms of work and energy stored in the system. The Lagrange equations are written in the form:

    \begin{equation}
        \tau = \frac{d}{dt}(\frac{\partial L}{\partial \dot{q}})^T - (\frac{\partial L}{\partial q})^T
        \label{langrange}
    \end{equation}

    where $\tau$ is the generalized forces on the system, $q$ is the joint vector, $\dot{q}$ represent the joint velocities and $L=K - U$ is the Lagrangian
    function with K as kinetic energy and U as potential energy respectively.

    The kinetic energy $K$ for body $i$ is expressed in the form,

    \begin{equation}
        K_i = \frac{1}{2} \int v^T_{m_i}v_{m_i}dm 
        \label{kinetic-energy}
    \end{equation}

    where $v_{m_i} = v_i + \omega_i \times \vec{r_{m_i}}$ is the velocity of point $m_i$ on the body $i$ which can be expressed as a function of twist, $\xi = [v_i^T, \omega_i^T]$

    The potential energy $U$ for body $i$ is defined as,

    \begin{equation}
        U_i = -m_i\mathbf{g}^T\vec{r_{CoM_i}}
    \end{equation}
    
    where $\mathbf{g}$ is the gravity vector and $\vec{r_{CoM_i}}$ is the distance vector to CoM of body $i$. The equation \ref{langrange} can be put in a matrix form as 

    \begin{equation}
        \tau = M(q).\ddot{q} + c(q, \dot{q})
        \label{eq: inverse-dynamic-model}
    \end{equation}

    where $M(q)$ is the generalized robot inertia matrix and $c(q, \dot{q})$ is the vector of Coriolis, centrifugal and gravity effects. This model is called as inverse dynamic model.

    \item \textbf{Newton Euler Formulation}
    
    Newton Euler (NE) equations allow computation of the sum of external forces $\sum f_j$ and moments $\sum m_{CoM_i}$ (including the gravity effects) is represented as

    \begin{align}
        \begin{split}
            \sum f_i &= m_i\dot{v_{CoM_i}} \\
            \sum m_{CoM_i} &= \mathbf{I}_{CoM_i}\dot{\omega_i} + \omega_i \times (I_{CoM_i}.\omega_i)
        \end{split}
        \label{eq: NE}
    \end{align}

    where $\dot{v_{CoM_i}}$ is the acceleration of the CoM of body $i$; $\dot{\omega_i}$ is the angular acceleration of the body $i$ and $\mathbf{I_{CoM_i}}$ is the inertia matrix of body $i$.
    These equations are usually recursive algorithms \cite{featherstone2014rigid,featherstone2000robot} which make the computation less expensive when a computer involves.

\end{enumerate}

The introduction of centroidal momentum and centroidal dynamics based on the Centre of Mass (CoM) of a system is particularly a import concept since the humanoids build large momenta(so far only
angular momentum is larger and linear momentum on legger humanoids are not upto the mark). The dynamic model of a robot can be expressed in two ways depending on the spaces that are used to describe the motion and the control input. The two approaches are: the joint space formulation, and the operational space 
(or task space) formulation. The joint space formulation is the classical approach and uses the joint space acceleration to specify the motion, and the generalized forces acting on the actuated joints 
to describe the control. The operational space formulation represents the motion directly using the task space acceleration, which needs a reformulation of the forces as task space generalized 
forces.

\subsubsection{Dynamic Control in Humanoid Robots}

For a humanoid robot, classical dynamic control techniques are not sufficient since coordination of the motion is required, environmental forces need to be considered, and balance has to be kept at all 
time. For the robot balance, a stability criterion such as keeping the CoM or the ZMP inside the support polygon must be enforced. For planar surfaces, the constraint on the 
ZMP implies a control of the contact forces. Thus, the interaction with the environment is always present since the feet are in contact with the ground and additional contacts of other parts of the robot 
body might also be necessary. These contacts generate an effect on the joints generalized forces, which cannot be controlled using only kinematic methods but with dynamic approaches. The dynamic control 
ensures the physical feasibility of the motion and allows for faster movements without losing balance.

The Operational Space Inverse Dynamics (OSID) is a more specific framework for controlling the whole-body of humanoid robots considering contacts and a set of different constraints. This approach
proposed in \cite{khatib2004whole,sentis2005control} is based on a two stage mapping to obtain consistent contact forces and is equivalent to successive projections onto the nullspaces of the previous tasks. Therefore, 
new tasks can be added without dynamically interfering with higher priority tasks. Other methods use the OSID within frameworks that 
involve some type of optimization to find the local solution. The most popular approaches use Quadratic Programming (QP), which allows for the specification of both equality and inequality constraints.
The latter type of constraints is fundamental in humanoid robotics to directly model unilateral contacts, and it is also important to properly specify some particular tasks. Although the previous approaches exploit the full robot dynamics, 
the angular momentum is not explicitly controlled within these frameworks. Nevertheless, it has been shown that the angular momentum 
is a natural and important part of human motion, specially when performing complex and fast movements \cite{popovic2004angular}. 

\subsection{Optimal Control}

Optimal control, also known in robotics as trajectory optimization or trajectory filtering, consists in finding a trajectory and its associated control law (policy) that satisfies some predefined 
optimality criterion. In general mathematical terms, it concerns the properties of control functions which, when inserted into a differential equation, give solutions that minimize a cost or measure
 of performance. But it also concerns optimization problems with dynamic constraints which might be functional differential equations, difference equations, partial differential equations or equations 
 with another form. This section is summarized from the literature work \cite{ramosponce} which provides a brief overview of optimal control in humanoid robots.


\subsubsection{Optimal Control in Humanoid Robots}

In robotics, optimal control can be used to find the trajectories from an initial posture to a final desired posture, specified as a whole or as a set of sub-objectives, satisfying certain constraints.
Very fast and powerful movements can be generated with optimal control, which can comprehend the problems of inverse kinematics or inverse dynamics and can therefore pro- duce better movements. The 
problem with IK and OSID alone is their inability to properly handle the CoM accelerations, thus overrestricting the motion. A solution typically relies on a dedicated submodel, like a linearized 
inverted pendulum \cite{kajita2003biped} to capture the future of the system, but this ad-hoc resolution increases the control architecture complexity. Thus, using these schemes the future states of the system
can be somehow predicted, but dedicated submodels would need to be developed for each case. However, optimal control is the most suitable approach that allows to take into account all the constraints
at the same time. In fact, optimal control can automatically generate the proper trajectory for the CoM in order to achieve fast movements: it acts as a classical pattern generator used for walking
schemes, but additionally incorporating whole-body motion. A serious challenge to optimization-based approaches in robotics is that the timescales of the dynamics are faster than in other applications 
and need a faster response. Currently, the main problem is the computational time, due to the high number of DoF, that forbids its use in real-time applications.

The main drawback of optimal control is the curse of dimensionality, which is particularly important for humanoid robots whose state space is so large that no control scheme can explore all of it in advance and 
prepare suitable responses for every situation. It would be desirable to obtain optimal control in real time; however, currently there is no approach that can achieve this: the solutions are very time consuming 
and generic solvers tend to get stuck into local minima or they even return trivial solutions. The problem of finding the proper formulation and resolution of optimal control is still an open issue in robotics.

\section{Approaches in Humanoid Balance Control}


There has been various control strategies that has been implemented on the humanoid robot for various tasks but, the balance control strategies are the dominant task when it comes to legged humanoids. 
The balance strategies are mostly based on the centroidal momentum and centroidal dynamics for the complex structured humanoids. These single point mass systems are less computational expensive considering
point approximation method of Centre of Mass (CoM). Note that the concepts of CoM and ZMP will be discussed later in chapter \ref{chapter-3}. The balance control approaches based on single point masses will be discussed in this section among which the most common methods are

\begin{enumerate}
    \item Linear Inverted Pendulum Model
    \item Double Inverted Pendulum Model
    \item Spherical Inverted Pendulum Model (Simple and Double)
\end{enumerate}


\subsection{Linear Inverse Pendulum Approach}
\label{sec: LIPM}
When it comes to real-time motion planning and control, single point masses plays extremely major role in holding the balance of the biped robots. A number of physical points can be simplified and summarized from 
these models where some of the points include Centre of Mass (CoM) and Centre of Pressure (CoP). The CoM is defined as the sum of the mass of the individual links of a system and CoP is defined as the point on the 
ground that represents the overall force interaction of all ground contact points. The dynamics of any system of rigid bodies can be approximated usind the dynamics at CoM. At any time $t$, the sum of the force acting 
on the system results in the acceleration of CoM in the system. Mathematically, it can be represented as,


\begin{equation}
    \label{eq: scaling-pendulum}
    S.\begin{bmatrix}
        f_{gr} \\
        \tau_{gr}
    \end{bmatrix} = 
    \begin{bmatrix}
        ma \\
        \dot{H}
    \end{bmatrix}
\end{equation}


where $m$ is the total mass of the system, $a$ is a $(3 \times 1)$ vector representing the CoM acceleration. $\dot{H}$ represent the change in amgular momentum, $f_{gr}$ is the ground reaction forces and $\tau_{gr}$ is the torque vector. And $S$ can be represented as the 
Selection Matrix equal to \cite{elhasairi2015humanoid}, 


\begin{equation}
    S = \begin{bmatrix}
        I_{3 \times 3} && I_{3 \times 3} && 0_{3 \times 3} && 0_{3 \times 3} \\
        (P_R - COM)\times && (P_L - COM)\times && I_{3 \times 3} && I_{3 \times 3}
    \end{bmatrix}
\end{equation}

where $P_R$ and $P_L$ are the feet position, $I_{N \times N}$ and $0_{N \times N}$ are the identity matrix and zero matrices of degree $N$; and $v\times$ is the left cross product matrix of any vector $v$. Equation \ref{eq: scaling-pendulum} 
represents the forces acting at CoM due to gravity and ground reaction forces in the first row and torque about CoM due to change in anguluar momentum.


\begin{figure}[h!]
    \centering
    \includegraphics[scale=.5]{images/LIP.jpg}\hfill
    \caption{Linear Inverse Pendulum Model}\hfill
    \label{ipm}
\end{figure}

Consider a simple inverted pendulum model as in figure [\ref{ipm}] with single point mass $m$ and link length $l$, given the assumptions that no angular momentum will be generated or changed for the single point mass. Furthermore, an additional assumption
is made to the dynamics of the system can be represented 
as,

\begin{equation}
    \label{eq: CoP}
    m\ddot{x} = \frac{m\mathbf{g}}{z_0}(x - x_{CoP})
\end{equation}

where $x$ represents the CoM horizontal position at a constant height $z_0$, and $x_{CoP} = \frac{F_T - \tau}{F_N}$ represent the CoP position where $F_T$ is the tangential component of the ground reaction force, $\tau$ is the ankle torque and $F_N = m\mathbf{g}$ is the normal
vertical component of the ground reaction force. Thus the LIPM model can be used for ankle balance strategy for legged and humanoid robots.

There are number of ways of using $X_{CoP}$ as the control input for the system as surveyed in \cite{elhasairi2015humanoid}.
Firstly, the rate of change of $x_{CoP}$ can be limited to achieve smooth trajectories. Secondly, the CoP position is often a measured quantity in most humanoid robots. Thirdly, it is not possible to change CoP position instantaneously. except during the 
transition from single to double support. 


\subsection{Double Inverse Pendulum Approach}
\label{sec: DIPM}

The single point mass based assumptions from the previous section \ref{sec: LIPM} can cause internal forces disturbing the system. This is due to the fact that the primary assumption is massless legs of the system.
These systems are linearised models of the good approximation of the system dynamics. 

Consider a double inverted pendulum model as in figure \ref{dipm} with mass $m$ and link lengths $l_1$ and $l_2$, to capture the effect of the upprt body motion of the biped balance, the humanoid robot can be modelled as 
non-linear fully-actuated, unconstrained double inverse pendulum mode. Then the equation of motion s can be given by,

\begin{equation}
    \label{eq: DIPM}
    M(q)\ddot{q} + c(q, \dot{q}) = \tau
\end{equation}

where $q$ is the joint vector, M is the mass-inertia matrix, $\tau$ is the torque vector and $c$ is the vector of a Corolis, centripedal and gravitational forces. The DIPM model used the hip-balance strategy defined in \cite{atkeson2007multiple}. In these models, the feet are normally
assumed to be in contact with ground and do not contribute kinetic or potential energy of the system. The CoP calculation is still the same assuming that it does not leave the support polygon defined by the feet as in equation \ref{eq: CoP}.
 
\begin{figure}[h!]
    \centering
    \includegraphics[scale=0.45]{images/DIPM.jpg}\hfill
    \caption{Double Inverse Pendulum Model}\hfill
    \label{dipm}
\end{figure}


\subsection{Spherical Inverse Pendulum Approach}

There are evidences to use simplified non-linear spherical inverse pendulum model for defining the motion of the humanoid robots. These models are marginally stable and if it is subjected to any slight disturbance, it will lose its upright stable posture.
To maintain this unstable equilibrium of the SIP model, different forms of feedback controls are used. The equations of motion for non-linear spherical inverse kinematic models are given by \cite{elhasairi2015humanoid}, 


\begin{align}
    \label{eq: non-linear SIPM}
    \begin{split}
    \ddot{\theta} &= \frac{\tau_\theta}{ml^2} + \frac{\mathbf{g}}{l}\cos(\phi)\sin(\theta) - \phi^2\sin(\theta)\cos(\theta) \\
    \ddot{\phi} &= \frac{\tau_\phi}{ml^2\cos^2(\theta)} + \frac{\mathbf{g}\sin*(\phi)}{lcos(\theta)} + 2\frac{\dot{\phi}\dot{\theta}\sin(\theta)}{\cos(\theta)}
    \end{split}
\end{align}

where $\theta$ and $\phi$ represent the ankle orientation along x- and z-axis respectively.
\begin{figure}[h!]
    \centering
    \includegraphics[scale=0.5]{images/SIPM.jpg}\hfill
    \caption{Linearized Spherical Inverse Pendulum Model}\hfill
    \label{fig: sipm}
\end{figure}

From equations \ref{eq: non-linear SIPM}, it can be seen that the system is non-linear and there is a degree of cross coupling between the two degrees of freedom. Like any other LIPM model, simple SIPM models use ankle-balance strategy. To use traditional control methods, the system needs to be linearized for small values of $\theta$
and $\phi$. Then the equation of motions describing the pendulum are represented as,


\begin{align}
    \label{eq: linear SIPM}
    \begin{split}
        \ddot{\theta} &= \frac{\tau_\theta}{ml^2} + \frac{\mathbf{g}}{l}\theta \\
        \ddot{\phi} &= \frac{\tau_\theta}{ml^2} + \frac{\mathbf{g}}{l}\phi
    \end{split}
\end{align}


Equations \ref{eq: linear SIPM} are the linearized approximation of the SIPM model which holds the dynamics of the system with $x_{CoP} = -l \times \sin(\theta)$ as,

\begin{equation}
    \label{eq: CoP SIPM}
    m\ddot{x} = \frac{m\mathbf{g}}{z_0}(x - x_{CoP})
\end{equation}

\subsection{Static vs Dynamic Balance}

During the walking gait, at any time, the projection of the centre of mass if the robot is within the support
polygon \cite{katic2003survey}. This allowed the static walker to maintain balance if stopped at any timestep.
However, this resulted in the slower walking speeds and larger translation of the centre of mass from one foot to
other. The effect of the inertial forces acting on the robot segments are not taken into account except for the 
gravitaional forces.

During dynamic balance, the centre of mass is allowed outside the support polygon to an extent but there might be 
parts where the robot might fall \cite{katic2003survey}. The velocity and acceleration of each link of the robot
model is taken into account resulting in comparitively faster motion maintaining the postural stability. Most of the
modern robots are using dynamic control for its efficiency and versatile gait structure compared to static control.


\section{Approaches in Dynamic Motion Retargetting}

This section presents the various works focusing on dynamics based motion retargetting systems implemented on real robots.

\subsection{Dyanmic Retargetting in HRP-2 robot}

The case study presented in \cite{ramosponce} for motion imitation focusing on dynamics of human motion allows quickly and 
efficiently to generate long sequence of dynamic movement for humanoid robots. The motion generation is based on combination 
of two very efficient tools: motion capture and hierarchical operational space inverse dynamics, which enables the retargetted 
dynamics to fit with the constraints of the real robot. The method produces reliable movements on the robot. 

\begin{figure}[h!]
    \centering
    \includegraphics[scale=0.25]{images/mansard's-work.png}\hfill
    \caption{Dynamic Retargetting Scheme \cite{ramosponce}}\hfill
    \label{fig: mansard's work}
\end{figure}

From this study, it can be considered that the operational-space inverse-dynamics is a mature tool that is able to replace 
the inverse kinematics approach for all robots where dynamics matters. Particularly, in humanoid robotics, inverse dynamics 
can be directly applied by sending the reference acceleration output, obtained with the inverse-dynamics solver, to the robot. 
In particular, the Stack of Tasks (SoT) proposed in \cite{ramosponce}, while enforcing the dynamics of the retargeted motion, 
provides the robot programmer with some easy edition capabilities to correct the defects of the retargeted motion or to augment
the original movement with some artificial features.

\subsection{Motion Retargetting in iCub}

The work proposed in \cite{gucci:hal-01895145} presebted a framework for teleoperation of iCub robot based on inverse kinematics 
with Quadratic solver. It allows a robust real-time retargetting of generic motions. The CoM retargetting and fall detection proposed
in this work is robust due to the presented ZMP correction approach that guarantees the stability of the retargeted motion in double support.
This approach is validated in simualtion and on real robot perfecting the balance restoration in the kinematic approach. More information
on this work is detailed and adapted in chapter \ref{chapter-3}.

\begin{figure}[h!]
    \centering
    \includegraphics[scale=0.4]{images/gucci-work.png}\hfill
    \caption{Motion Retargetting Pipeline \cite{gucci:hal-01895145}}\hfill
    \label{fig: gucci's work}
\end{figure}

\subsection{Hybrid Retargetting using Time Scaling}

This work by Kumar \cite{karthikmunirathinam} is to take advantage of motion tracking by manipulating the motion by time scaling. This method changes
the traveling time of the reference motion by decelerating the motion if unbalanced and by accelerating the motion in order to synchronise with the
reference motion by decelerating the motion if unbalanced and by accelerating inorder to keep it synchronising with the human actor. This approach serves as 
a different variant in the domain of motion imitation of humanoid robots since when a suitable solution with time scaling is not found, joint based control 
to ensure the balance of the system is turned back on. A set of motions has been recorded and are validated in simuation using virtual NAO robot. 