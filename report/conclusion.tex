\chapter*{Conclusion}
\addcontentsline{toc}{chapter}{Conclusion}


Humanoid robots are redundant open serial chains not fixed to their environment with a few 
dozen DOF. In order to cope with this complexity, the approach of imitation uses captured 
human motion as the basis for motion generation. This approach takes much of the burden of 
motion prediction away from the robot.

Due to the kinematic and dynamic differences between human beings and humanoid robots, motion 
transfer is not a straightforward matter. Besides taking into account the physical limitations
 of the robot, issues such as kinematic singularities, self-collision and balance must be 
 addressed. Moreover, for real-time imitation, time constraints must be enforced.


In this research, whole-body motion imitation of human motion by the humanoid robot NAO 
using hierrachical quadratic programming was implemented with five tasks: tracking 
end-effectors, keeping static balance, avoiding joint limits, posture control and tracking human joint values. 

A method to scale the human motion to the robot’s dimensions one body segment at a time was introduced. 
The scaled end-effector positions in Cartesian space were tracked while keeping static balance for single 
support on either foot, double support and transition between supports. The joint limits were avoided 
by a combination of a clamping loop and a task which minimizes the distance from the average joint value. 
The joint values were kept as close as possible to the performer’s whenever that didn’t affect the higher 
priority tasks to imitate the manner as well as possible. More information and discussion will be added on the 
final version of the report.
